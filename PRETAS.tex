\raggedright

{\Formular{\textbf{FERNAND DELIGNY} \textit{nasceu no norte da França em 1913. Conhecido
como educador, preferia autonomear"-se poeta e etólogo: “Meu projeto
era escrever”. A escrita é nele uma atividade existencial, o
laboratório permanente da sua prática. Dedica 50 anos da vida a crianças
inadaptadas, delinquentes, psicóticos e autistas. Etólogo designa o meio
que ele reinventa, em todas as suas circunstâncias, para tentar dar a
essas crianças a oportunidade de sobreviver em uma comunidade que exclui
ou normaliza.}

\textit{Deligny publicou 15 livros, mais de cem artigos, fez filmes, fotografou e
escreveu milhares de páginas inéditas: novelas inacabadas, roteiros,
contos, ensaios infinitamente recomeçados sobre o humano, a linguagem, o
psiquiátrico, a rede, o aracnídeo; esses dois últimos termos são marcas
de seu glossário. Dois de seus livros, {\textsl\textsc{Os vagabundos eficazes}} e
{\textsl\textsc{O aracniano}}, estão publicados pela n-1.}
}}




